\documentclass[a4paper,11pt]{article}
\usepackage[T2A]{fontenc}
\usepackage[utf8]{inputenc}
\usepackage{csquotes}
\usepackage{amsthm}
\usepackage{amsmath}
\usepackage{float}
\usepackage{tikz}
\usetikzlibrary{decorations.pathreplacing}
\usetikzlibrary{arrows.meta}
\usetikzlibrary{angles, arrows.meta, decorations.pathmorphing}
\usetikzlibrary{fixedpointarithmetic}
\usetikzlibrary{patterns}
\usepackage{fp}
\usepackage{adjustbox}
\usepackage{pgfplots} % Подключаем пакет
\pgfplotsset{compat=1.18} % Р
\usepackage{wrapfig}
\usepackage[english,russian]{babel}
\usepackage{booktabs} % Для красивых таблиц
\usepackage{multirow} % Для объединения строк
\usepackage{adjustbox}
\usepackage{soul} % Основной пакет для подчеркивания
\usepackage{xcolor}
\usepackage{stackengine}
\usepackage{ulem}
\setul{0.5ex}{0.3ex}
\setulcolor{gray}
\usepackage{dashrule}
\usepackage{blindtext}
\usepackage{epigraph} 
\usepackage{csquotes}

\tolerance=1
\emergencystretch=\maxdimen
\hyphenpenalty=10000
\hbadness=10000

\begin{document}
\begin{center}
Транснациональные корпорации
\end{center}
\section{Введение}
\begin{quotation}
\itshape
Финансовый капитал хочет не свободы, а господства
\end{quotation}
Главной силой глобального процесса интернационализации являются транснациональные корпорации (ТНК), представляющие наиболее мощную часть международного корпоративного бизнеса, действующие в глобальных масштабах и играющие ведущую роль в укреплении мирохозяйственных связей, а также в определении направления развития мирохозяйственной системы в целом.\\
Экономическая реальность такова, что стратегия развития национальных экономик большинства стран полностью определяется (хотят они этого или не хотят) характером и интенсивностью функционирования современных транснациональных корпораций, распоряжающихся средствами, превышающими размеры национального дохода многих суверенных государств. \\
Вся послевоенная экономическая история западноевропейских стран, Японии и новых индустриальных стран свидетельствует о том, что национальный капитал способен выдержать конкуренцию с ними лишь в том случае, если он сам структурируется в достаточно мощные финансово-промышленные образования, адекватные международным аналогам и способные проводить самостоятельную и активную внешнеэкономическую политику.

\section{Факторы интернационализации бизнеса}
\begin{quotation}
\itshape
Капиталисты делят мир не по своей особой злобности, а потому, что достигнутая ступень концентрации заставляет становиться на этот путь для получения прибыли
\end{quotation}
Интернационализацией бизнеса затронуты все более или менее значимые компании мирового хозяйства.\\
Какие же факторы заставляют конкретную компанию принимать решение об интернационализации своего бизнеса? \\
\subsection{Внутрифирменные факторы}

\textbf{Поиск новых рынков сбыта} - Многие компании идут на интернационализацию своего бизнеса в поисках новых, более емких рынков сбыта. Наиболее выгодны в этом смысле рынки стран со значительным населением: Китай, Индия, Бразилия (при прочих равных условиях). Однако при принятии такого рода решений не надо путать число жителей конкретной страны с количеством потенциальных покупателей продукции компании, определяющих соответствующий платежеспособный спрос на эту продукцию. \\
\textbf{Поиск необходимого сырья} - именно поэтому большинство нефтеперерабатывающих компаний инвестировали значительные средства в страны Ближнего Востока, а многие предприятия, связанные с текстильным производством, а также известные Дома мод открыли свои производства в Индии и Пакистане — странах-производителях первоклассного хлопка-сырца.\\
\textbf{Поиск условий более эффективного производства} - Достаточно частой причиной интернационализации бизнеса является стремление компаний повысить эффективность производства и (или) распределения своей продукции. Именно по этой причине Philips переместил свое производство в Малайзию и Сингапур, годами специализирующиеся на эффективном производстве электроники.\\
\textbf{Успех в национальной экономике} - Если компания смогла добиться значительного успеха на местном рынке, то ее дальнейший рост является, как правило, совершенно естественным. В этом случае компания может с уверенностью выходить на мировой рынок, тем более что в современных условиях большинство товаров пользуются примерно одинаковым спросом во всех стран мира.
\subsection{Факторы внешней среды}
\textbf{Зарубежные бизнес-предложения} - При поступлении таких предложений компания оценивает их по критериям потенциалов соответствующих рынков и рисков, с ними связанных. Начав исследовать зарубежный рынок, компания постепенно все больше и больше привыкает к мысли о возможности своей работы на нем.\\
\textbf{Соревновательный фактор} -  Так как большинство производств в настоящее время носят олигополистический характер, независимые действия какой-либо компании вызывают неопределенность, связанную с конкурентными позициями олигополистов. Как только такая компания принимает решение об интернационализации и работе на зарубежном рынке, конкуренты, как правило, делают то же самое, так как опасаются ухудшения своих конкурентных позиций.\\
\textbf{Сильная зарубежная конкуренция} - сильная зарубежная конкуренция на национальном рынке. Фирмам приходится защищаться, и во многих случаях это приносит им только пользу, так как они начинают работать более активно и находить новые, ранее не охваченные сегменты рынка.

\section{Методы интернационализации бизнеса}
\begin{quotation}
\itshape
Для старого капитализма, с полным господством свободной конкуренции, типичен был вывоз товаров. Для новейшего капитализма, с господством монополий, типичным стал вывоз капитала.
\end{quotation}
Рассмотрев причины, по которым компания интернационализирует свой бизнес, уместно задаться вопросом: какие методы могут быть использованы конкретным предприятием для проникновения на зарубежный рынок?\\
\textbf{Экспорт} -  самая традиционная и наиболее простая форма ведения международного бизнеса, подразделяемая обычно на косвенный и прямой экспорт.\\
В первом случае экспортные функции выполняет специальная компания, от качества работы которой во многом зависит имидж продукции экспортера. Объемы продаж с использованием такой формы экспорта весьма незначительны.\\
Во втором случае компания экспортирует собственную продукцию, задействуя для этого коммерческие филиалы, представителей компании, дистрибьюторов, что позволяет ей оперативно отслеживать конъюнктуру и адаптировать свою продукцию к нуждам иностранного потребителя.\\
\textbf{Передача прав на интеллектуальную собственность} - Интеллектуальная собственность является важным коммерческим активом международных компаний и во многих случаях представляет существенный, если не основной, источник конкурентоспособности компании. Масштаб и возможности лицензирования деятельности особенно важны в международном бизнесе.\\
\textbf{Лицензирование} - Лицензии в зависимости от объёма уступаемых прав могут быть простыми, исключительными и полными.\\
При \textbf{исключительной лицензии} покупателю предоставляется исключительное право на использование объекта лицензии на согласованной территории.\\
\textbf{Полная лицензия} означает передачу компании, приобретающей её, всех прав на использование объекта лицензии, т. е. фактически означает полную продажу интеллектуальной собственности.\\
При \textbf{простой лицензии} продавец лицензии либо оставляет за собой право использования объекта лицензии на оговоренной территории, либо уступает это право другим фирмам.\\
\textbf{Франчайзинг} - это форма ведения бизнеса, при которой компания-франчайзер передает компании-покупателю право на использование своего товарного знака и оказывает ей постоянную поддержку в ведении бизнеса.\\
\textbf{Прямые иностранные инвестиции} - то основной способ для международной компании повысить свою конкурентоспособность в условиях современной глобальной экономики. На практике международная компания может осуществлять прямые иностранные инвестиции в следующих формах:
\begin{itemize}
\item приобретение существующей иностранной компании;
\item создание совместного предприятия в форме акционерного общества на зарубежном рынке;
\item создание компаний за рубежом «с нуля».
\end{itemize}

\section{Эволюция транснациональных компаний}
\begin{quotation}
\itshape
Концентрация производства; монополии, вырастающие из нее; слияние или сращивание банков с промышленностью – вот история возникновения финансового капитала и содержание этого понятия.
\end{quotation}
Международное производство и в общем случае международный бизнес базируются, на процессах интернационализации производства и капитала, т. е. создания тесных кооперационных связей и переплетения капиталов в разных странах в форме прямых и портфельных инвестиций. В результате у некоторых, прежде всего, крупных компаний появляются хозяйственные структуры (филиалы) в других странах, осуществляющие вместе с головной (материнской) компанией общую экономическую деятельность. Превалирующей в рамках таких ТНК является производственная функция, определяющая уровень конкурентоспособности их товаров и услуг на мировом рынке.\\
В процессе своего становления современные ТНК претерпели несколько эволюционных трансформаций, которые условно можно разделить на следующие этапы, или поколения.
\subsection{Этап I (до 1920 г.)}
Деятельность ТНК первого поколения в значительной мере была связана с разработкой сырьевых ресурсов, что дает основание определить их как колониально-сырьевые ТНК — например, British Petrolium (BP) (1909), US Steel (1904).\\ В результате своего функционирования сырьевые ТНК поставили под жесткий контроль рынки целого ряда стран. Вместе с тем при посредстве своих зарубежных филиалов они стремились расширить рынки сбыта собственной продукции. В первой половине XX в. все страны мира защищали свои внутренние рынки весьма высокими таможенными барьерами, преодолевать которые внешним поставщикам товаров и услуг было весьма трудно. Гораздо проще было обойти их путем создания в той или иной стране филиала компании-экспортера. Находясь внутри защищенного рынка, такой филиал мог сбывать продукцию компании практически без таможенных пошлин.
\subsection{Этап II (1920—1949)}
ТНК второго поколения стали широко использовать стратегию вывоза промышленного капитала, где особенно преуспели мощные в финансовом и техническом отношении монополии США. К тому же рассмотренный период характеризуется ростом числа новых ТНК и постепенным вовлечением местных фирм в международное производство, что создало предпосылки для дальнейших слияний и поглощений, а также образования большого числа стратегических альянсов.
\subsection{Этап III (1950—1979)}
Особенно активные действия по завоеванию мировых рынков ТНК начали с начала 1950-х гг., чему способствовали политика либерализации международных экономических связей, появление новых освободившихся от колониализма государств, растущий потребительский спрос и другие факторы. Бурное развитие ТНК как по количеству, так и по масштабам и объемам деятельности способствовали тому, что они стали играть одну из важнейших ролей в международной экономике. Отличительными особенностями ТНК третьего поколения стали расширение и интенсификация их деятельности в сфере услуг.
\subsection{Этап IV (1980—1995)}
Рассматриваемый этап развития ТНК характеризуется широким использованием такой формы прямых иностранных инвестиций, как стратегические альянсы, слияния и поглощения преимущественно производственных компаний. \\ Исключительно важную роль в механизме функционирования глобальных ТНК играют банковские и финансовые институты. Процессы интернационализации и глобализации мировой экономики, острая конкурентная борьба в финансовой сфере способствовали формированию и развертыванию деятельности транснациональных банков (ТНБ). Становится нормой, когда обязательным элементом в составе ТНК являются мощные финансовые институты, осуществляющие в международном масштабе операции по поглощению и слиянию с другими компаниями, лизингу, кредитованию и инвестированию.\\
Глобальные ТНК последовательно проводят стратегию образования крупных групп, объединяющих производственные, торговые и финансовые компании. Однако помимо экономических альянсов крупнейших ТНК между собой, глобальные ТНК усиливают взаимодействие со средним и малым бизнесом как в стране базирования, так и с зарубежными партнерами. В частности, расширяется сеть субподрядных связей ТНК с мелкими фирмами, очень успешно выполняющими функции как по поискам и внедрению изобретений, так и по самостоятельной разработке новой продукции и техники. Развитие разветвленной системы субподряда позволяет ТНК освободиться от многих нерентабельных для них операций и функций и сосредоточиться на наиболее прибыльных и перспективных сферах деятельности. Вместе с тем глобальные ТНК, заинтересованные в получении от своих подрядчиков добротных полуфабрикатов или комплектующих, вынуждены передавать им современные технологии. Обеспечивая таким образом высокотехнологичную основу мелкого производства, глобальные ТНК способствуют его транснационализации.
\subsection{Этап V (1995—2005)}
ТНК пятого поколения по своей финансовой и экономической мощи стали независимыми субъектами в мировой экономике наряду с национальными экономическими системами и соответствующими им государствами. Для ТНК пятого поколения характерен преимущественный рост капиталовложений в сферу услуг. В частности, более 50\% прямых иностранных инвестиций крупнейших ТНК с середины 1990-х гг. направлялись в данный сектор экономики.
\subsection{Этап VI (2005 — н. вр.)}
Существенным аспектом деятельности шестого поколения ТНК является интеграция транснациональных банков (ТНБ), что приводит к возникновению огромных транснациональных финансовых конгломератов и еще более усиливает позиции ТНК в мирохозяйственной системе.\\
Результатом многочисленных слияний и поглощений, а также стратегических альянсов становится образование новых ТНК олигополистического типа. В настоящее время в развитых странах доминирующее положение в каждой отрасли занимают всего два-три супергиганта, конкурирующие между собой на рынках всех стран.

\section{Структура транснациональной компании}
\begin{quotation}
\itshape
Несомненен тот факт, что переход капитализма к ступени монополистического капитализма, к финансовому капиталу связан с обострением борьбы за раздел мира.
\end{quotation}
ТНК представляют собой экономические объединения фирм на базе единого титула собственности, принадлежащего материнской компании, которая контролирует зарубежные активы родственных предприятий посредством владения определенной частью их капитала.
\textbf{Материнская компания} регистрируется как юридическое лицо и является головной компанией ТНК.
\begin{itemize}
\item разрабатывает общие направления и конкретные цели функционирования и развития всей ТНК;
\item определяет средства, формы и методы достижения этих целей;
\item осуществляет контроль за выполнением своих установок и вносит в них коррективы;
\item контролирует финансовую деятельность всех подразделений путем составления консолидированного баланса, предоставляемого акционерам ТНК.
\end{itemize}
\textbf{Филиал ТНК} не имеет юридической самостоятельности и, следовательно, не может вести дела от своего имени: он действует по поручению материнской компании и, как правило, имеет одинаковое с ней наименование. В обязанности зарубежного производственного филиала входят чаще всего выпуск тех видов продукции, в которых заинтересована материнская компания, и реализация их на тех рынках, которые она определит.\\
\textbf{Дочерние компании} в отличие от филиалов являются юридически самостоятельными. Заключение сделок и ведение финансовой документации осуществляются ими отдельно от материнской компании, которая не несет при этом никакой ответственности по обязательствам дочерней компании. Вместе с тем материнская компания всегда осуществляет необходимый контроль за деятельностью принадлежащих ей дочерних компаний, возможность которого обеспечивается владением контрольными пакетами акций своих «дочек».\\
\textbf{Совместным предприятием} в практике международного бизнеса называют фирму с участием одного или нескольких иностранных партнеров-инвесторов. \\Вот несколько очевидных преимуществ сотрудничества с таким партнером:
\begin{itemize}
\item местный партнер лучше знает и понимает покупателей, традиции, общественные отношения конкретной страны;
\item контакты и репутация местных партнеров расширяют доступ к рынку капитала принимающей страны;
\item местный партнер может располагать подходящей для данной среды технологией, которая, вероятно, может быть использована по всему миру;
\item если принимающая страна требует, чтобы иностранные фирмы делили право собственности с местными фирмами или инвесторами, то 100\%-ное иностранное владение становится просто нереальным, и единственно возможной формой международного бизнеса становится совместное предприятие;
\end{itemize}
По типу взаимоотношений материнской компании и ее зарубежных дочерних предприятий все ТНК могут быть подразделены на три типа:
\begin{itemize}
\item этноцентрические;
\item полицентрические;
\item геоцентрические;
\end{itemize}
\textbf{Этноцентрический тип} характеризуется последовательной ориентацией высшего руководства ТНК на приоритет материнской компании. При этом типе зарубежные рынки остаются для корпорации, прежде всего, продолжением внутреннего рынка страны базирования материнской компании. ТНК в этом случае создают иностранные филиалы главным образом для обеспечения себе надежных поставок дешевого сырья или для обеспечения зарубежных рынков сбыта.\\
\textbf{Полицентрический тип} характеризуется возрастанием значимости внешнего рынка, когда внешний рынок становится зачастую даже более важным сектором деятельности ТНК, нежели рынок внутренний. У таких ТНК зарубежные филиалы крупнее и разнообразнее, они не столько продают продукцию материнской компании, сколько производят ее на месте в соответствии с потребностями их рынков.\\
\textbf{Геоцентрический подход} характерен для наиболее зрелого типа ТНК. Эти ТНК являются как бы децентрализованной федерацией региональных филиалов. Материнская компания рассматривает себя здесь не как центр ТНК, а лишь как одну из ее частей. Ареной деятельности геоцентрической ТНК является весь мир.

\section{Роль транснациональных компаний в мировой экономике}
\begin{quotation}
\itshape
Финансовый капитал – такая крупная, можно сказать, решающая сила во всех экономических и во всех международных отношениях, что он способен подчинять себе и в действительности подчиняет даже государства, пользующиеся полнейшей политической независимостью.
\end{quotation}
Современные ТНК, как отмечалось, играют в мировой экономике доминирующую роль. Признанием этого стало создание в 1974 г. специальной комиссии ООН по изучению феномена ТНК, которая впоследствии вошла в Конференцию ООН по торговле и развитию (ЮНКТАД), материалы исследований и статистика которой являются основным официальным источником по развитию ТНК.\\

% Надо будет вставить в презентацию табличку (или ее часть) с Крупнейшие компании мира 

В начале XXI в. в мире насчитывается более 85 тыс. ТНК и 900 тыс. их филиалов. Материнские компании расположены главным образом в развитых странах (50,2 тыс.), большее число филиалов приходится на развивающиеся страны. \\
В настоящее время ТНК контролируют более 50\% мирового промышленного производства, свыше 2/3 мировой внешней торговли, а также около 80\% мировой базы патентов и лицензий на новую технику, технологий и ноу-хау. На долю ТНК приходится большая часть прямых зарубежных инвестиций в мире 90\%, причем первая сотня самых крупных ТНК концентрирует в своих руках около 40\% их общемирового объема. Под контролем ТНК находится 90\% мирового рынка пшеницы, кофе, кукурузы, лесоматериалов, табака, джута и железной руды; 85\% рынка меди и бокситов; 80\% — чая и олова; 75\% — бананов, натурального каучука и сырой нефти.
На предприятиях ТНК работает более 70 млн чел.; с учетом смежных отраслей и инфраструктуры ТНК обеспечивают работой около 150 млн чел.
Совокупные валютные резервы ТНК в 5—6 раз превышают резервы центральных банков всех стран мирового хозяйства.
\subsection{Влияние транснациональных компаний на экономику стран}
Существуют две сферы отношений ТНК с государствами и правительствами: одна - между материнской структурой международной корпорации и правительством ее родины (страны базирования); и другая - между ТНК и правительством принимающей страны.\\
В эпоху, когда прямой политический контроль над территориями других государств стал невозможен, ТНК позволяют «своим» странам, имеющим технологическое и экономическое преимущество, получать беспрепятственный доступ к ресурсам других государств. Кроме того, в тех случаях, когда другие государства могут применить протекционистскую политику в отношении товаров экономически развитых стран, ТНК, размещая производство на территории этих государств, избавляют «свои» страны от необходимости преодолевать протекционистские барьеры и тем самым сохраняют за ними данное рыночное пространство в рамках существующей системы международного разделения труда. Образуя на территории других государств анклавы своей собственности в виде филиалов или дочерних компаний, ТНК создают и укрепляют за рубежом позиции своей страны базирования.Новой тенденцией в отношениях ТНК со страной базирования является стремление правительств некоторых стран установить над ними определенный государственный контроль, что привело к появлению довольно многочисленной группы транснациональных корпораций с государственным участием.\\
Сложнее складываются отношения ТНК с менее развитыми, в том числе развивающимися странами, что обусловлено частым несовпадением их экономических интересов. В целом ряде случаев ТНК могут внести определенный вклад в развитие рассматриваемой группы стран, что находит отражение:
\begin{itemize}
\item в получении ими передовых производственных технологий в некоторых отраслях национального производства;
\item содействии превращению развивающихся стран из экспортеров сырья в продавцов готовой продукции;
\item увеличении занятости местного населения и роста доходов местных работников корпораций;
\end{itemize}

Однако практика взаимоотношений ТНК с развивающимися странами демонстрирует также и значительное число фактов негативного влияния транснационального капитала на экономику соответствующих принимающих государств. Среди основных причин подобного влияния можно назвать следующие:
\begin{itemize}
\item возможность жесткой конкуренции со стороны ТНК местным компаниям;
\item опасность превращения принимающей страны в место сброса устаревших и экологически опасных технологий;
\item захват иностранными фирмами наиболее развитых и перспективных сегментов промышленного производства и научно-исследовательских структур принимающей страны;
\item противодействие реализации экономической политики тех государств, где ТНК осуществляет свою деятельность;
\item нарушение законодательства страны пребывания. Так, манипулируя политикой трансфертных цен, дочерние компании ТНК обходят национальные законодательства, укрывая доходы от налогообложения путем перекачивания их из одной страны в другую;
\item установление монопольных цен, диктат условий, ущемляющий интересы развивающихся стран;
\end{itemize}

\end{document}